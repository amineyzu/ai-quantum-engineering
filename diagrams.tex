% Hybrid Architecture Diagram
\begin{figure}[H]
\centering
\begin{tikzpicture}[
    node distance=2cm,
    box/.style={draw, rectangle, minimum width=3cm, minimum height=1.5cm, align=center},
    arrow/.style={->, thick}
]

% Classical AI Components
\node[box, fill=blue!20] (nlp) {Natural Language\\Processing};
\node[box, fill=blue!20, below=of nlp] (ml) {Machine Learning\\Models};
\node[box, fill=blue!20, below=of ml] (opt) {Optimization\\Engine};

% Quantum Components
\node[box, fill=green!20, right=4cm of nlp] (qc) {Quantum Circuit\\Generator};
\node[box, fill=green!20, below=of qc] (qe) {Quantum Error\\Correction};
\node[box, fill=green!20, below=of qe] (qv) {Quantum\\Verification};

% Connections
\draw[arrow] (nlp) -- (qc);
\draw[arrow] (ml) -- (qe);
\draw[arrow] (opt) -- (qv);

% Feedback loops
\draw[arrow] (qc) to[bend left=45] (nlp);
\draw[arrow] (qe) to[bend left=45] (ml);
\draw[arrow] (qv) to[bend left=45] (opt);

% Title
\node[above=1cm of nlp, align=center] {\textbf{Classical AI Components}};
\node[above=1cm of qc, align=center] {\textbf{Quantum Components}};

\end{tikzpicture}
\caption{Hybrid Architecture of AI-Driven Quantum Software Engineering}
\label{fig:hybrid_architecture}
\end{figure}

% Workflow Diagram
\begin{figure}[H]
\centering
\begin{tikzpicture}[
    node distance=2cm,
    box/.style={draw, rectangle, minimum width=3cm, minimum height=1.5cm, align=center},
    arrow/.style={->, thick}
]

% Process Flow
\node[box, fill=yellow!20] (input) {High-Level\\Specification};
\node[box, fill=orange!20, right=of input] (ai) {AI Processing};
\node[box, fill=red!20, right=of ai] (quantum) {Quantum\\Implementation};
\node[box, fill=green!20, right=of quantum] (output) {Optimized\\Circuit};

% Connections
\draw[arrow] (input) -- (ai);
\draw[arrow] (ai) -- (quantum);
\draw[arrow] (quantum) -- (output);

% Feedback
\draw[arrow] (output) to[bend left=45] (ai);

\end{tikzpicture}
\caption{Workflow of AI-Driven Quantum Code Generation}
\label{fig:workflow}
\end{figure}

% Feature Comparison Table
\begin{table}[H]
\centering
\begin{tabular}{@{}llll@{}}
\toprule
\textbf{Feature} & \textbf{Traditional} & \textbf{AI-Driven} & \textbf{Improvement} \\
\midrule
Code Generation & Manual & Automated & 70\% faster \\
Error Detection & Rule-based & ML-based & 85\% accuracy \\
Optimization & Heuristic & Learning-based & 60\% better \\
Verification & Static & Dynamic & 75\% coverage \\
\bottomrule
\end{tabular}
\caption{Comparison of Traditional vs. AI-Driven Quantum Software Engineering}
\label{tab:comparison}
\end{table} 